\documentclass{beamer}

%\usepackage{beamerthemesplit}
%\usetheme{Madrid}
%\usepackage{pgf}
\usecolortheme{orchid}
\usepackage{pgf}

%\hyphenpenalty=5000
%\tolerance=1000

\title{Decision Trees and Industrial Machine Learning}
\author{Jeremy Barnes, Recoset \\ (jeremy@recoset.com)}
\date{\today}

\begin{document}

\begin{frame}
  \titlepage
  \begin{center}
    www.recoset.com \\
  \end{center}
\end{frame}

\begin{frame}{Contents}
  \begin{itemize}
  \item Decision Trees
  \item Industrial Machine Learning
  \item My toolbox
  \end{itemize}
\end{frame}

\pgfdeclareimage[interpolate=true,height=7cm]{chess-8x8-10}{chess-8x8-10}
\pgfdeclareimage[interpolate=true,height=7cm]{chess-8x8-100}{chess-8x8-100}
\pgfdeclareimage[interpolate=true,height=7cm]{chess-8x8-1000}{chess-8x8-1000}
\pgfdeclareimage[interpolate=true,height=7cm]{chess-8x8-10000}{chess-8x8-10000}
\pgfdeclareimage[interpolate=true,height=3.5cm]{sm-chess-8x8-10000}{chess-8x8-10000}
\pgfdeclareimage[interpolate=true,height=3.5cm]{sm-chess-8x8-tr-10000}{chess-8x8-tr-10000}

\begin{frame}

  \begin{center}
    \pgfuseimage{chess-8x8-10}
  \end{center}

\end{frame}

\begin{frame}

  \begin{center}
    \pgfuseimage{chess-8x8-100}
  \end{center}

\end{frame}

\begin{frame}

  \begin{center}
    \pgfuseimage{chess-8x8-1000}
  \end{center}

\end{frame}

\begin{frame}

  \begin{center}
    \pgfuseimage{chess-8x8-10000}
  \end{center}

\end{frame}

\begin{frame}

  \begin{center}
    \begin{tabular}{ccc}
      \pgfuseimage{sm-chess-8x8-10000} & $\Rightarrow$ &
      \pgfuseimage{sm-chess-8x8-tr-10000}
    \end{tabular}
  \end{center}

\end{frame}

\pgfdeclareimage[interpolate=true,height=7cm]{ring-8-10}{ring-8-10}
\pgfdeclareimage[interpolate=true,height=7cm]{ring-8-100}{ring-8-100}
\pgfdeclareimage[interpolate=true,height=7cm]{ring-8-1000}{ring-8-1000}
\pgfdeclareimage[interpolate=true,height=7cm]{ring-8-10000}{ring-8-10000}
\pgfdeclareimage[interpolate=true,height=3.5cm]{sm-ring-8-10000}{ring-8-10000}
\pgfdeclareimage[interpolate=true,height=3.5cm]{sm-ring-8-tr-10000}{ring-8-tr-10000}
\pgfdeclareimage[interpolate=true,height=3.5cm]{sm-ring-8-tr2-10000}{ring-8-tr2-10000}

\begin{frame}

  \begin{center}
    \pgfuseimage{ring-8-10}
  \end{center}

\end{frame}

\begin{frame}

  \begin{center}
    \pgfuseimage{ring-8-100}
  \end{center}

\end{frame}

\begin{frame}

  \begin{center}
    \pgfuseimage{ring-8-1000}
  \end{center}

\end{frame}

\begin{frame}

  \begin{center}
    \pgfuseimage{ring-8-10000}
  \end{center}

\end{frame}

\begin{frame}

  \begin{center}
    \begin{tabular}{ccc}
      \pgfuseimage{sm-ring-8-10000} & $\Rightarrow$ &
      \pgfuseimage{sm-ring-8-tr-10000} \\
      & & \pgfuseimage{sm-ring-8-tr2-10000}
    \end{tabular}
  \end{center}

\end{frame}


\pgfdeclareimage[interpolate=true,height=3.5cm]{tank-yes}{tank-yes}
\pgfdeclareimage[interpolate=true,height=3.5cm]{tank-no}{tank-no}

\begin{frame}{Story: Tanks in the Desert}

  \begin{center}
    \begin{tabular}{cc}
      \pgfuseimage{tank-yes} & \pgfuseimage{tank-no} \\
    \end{tabular}
  \end{center}
  
\end{frame}

\pgfdeclareimage[interpolate=true,height=7cm]{insulin}{insulin}

\begin{frame}{Story: Medical Expense Fraud}

  \begin{center}
    \pgfuseimage{insulin}
  \end{center}
  
\end{frame}

\pgfdeclareimage[interpolate=true,height=7cm]{notaterrorist}{Khalid_bin_Sultan}

\begin{frame}{Story: Named Entity Classification}

  \begin{center}
    \pgfuseimage{notaterrorist} \\
    \begin{tiny}{Khalid bin Sultan, son of the Saudi Crown Prince, with US Army Commander Norman Schwartzkopf}\end{tiny}
  \end{center}
  
\end{frame}


\begin{frame}{Explaining a Forest of Decision Trees}

  \begin{columns}
    \begin{column}{0.35 \textwidth}
      \begin{centering}
      \pgfdeclareimage[interpolate=true,height=3.5cm]{decision-tree-explain}{decision-tree-explain}
      \structure{P(car crash)}
      \vspace*{\fill}
      \pgfuseimage{decision-tree-explain}
      \small \vfill
      \begin{tabular}{|l|r|r|} \hline
        \structure{Feature} & \structure{Value} & \structure{Score} \\
        \hline
        Bias &    &  0.3 \\
        Male & Y  & +0.1 \\
        Age  & 45 & -0.3 \\
        \hline
      \end{tabular}
      \end{centering}
    \end{column}
    \begin{column}{0.65 \textwidth}
      \begin{itemize}
      \item Immensely useful in finding problems
      \item Given a \alert{feature vector}, what features were the \alert{strongest contributors} to the result?
      \item Record the prediction at both \alert{internal nodes} and \alert{leaves}
      \item When we follow a branch, we assign the difference between the node predictions to the splitting feature
      \item For multiple trees, we take the weighted sum
      \end{itemize}
    \end{column}
  \end{columns}
\end{frame}

\begin{frame}{Boosting Algorithm}
  \pgfdeclareimage[interpolate=true,height=6cm]{boosting}{boosting}
  \pgfdeclareimage[interpolate=true,height=1.5cm]{boosting1}{boosting1}
  \pgfdeclareimage[interpolate=true,height=1.5cm]{boosting4}{boosting4}
  \pgfdeclareimage[interpolate=true,height=1.5cm]{boosting8}{boosting8}
  \pgfdeclareimage[interpolate=true,height=1.5cm]{boosting64}{boosting64}

  \begin{columns}
    \column[T]{.1\textwidth}
    \pgfuseimage{boosting1}\\
    \pgfuseimage{boosting4}\\
    \pgfuseimage{boosting8}\\
    \pgfuseimage{boosting64}

    \column[T]{.8\textwidth}
    \begin{itemize}
    \item Leverages a \alert{weak} and \alert{low-capacity} learning algorithm to produce state of the art accuracy
      \begin{itemize}
      \item The weak learning algorithm might have characteristics we want to preserve
      \end{itemize}
    \item Runs the \alert{weak learner} multiple times with \alert{extra weight on difficult examples}
    \item Produces a linear combination (\alert{ensemble}) of weak classifiers
    \item Adds \alert{automatic capacity control}
    \item Makes weak learner \alert{much more aggressive}
    \end{itemize}
    
  \end{columns}
\end{frame}

\begin{frame}

  \pgfdeclareimage[interpolate=true,height=3cm]{boosting1}{boosting1}
  \pgfdeclareimage[interpolate=true,height=3cm]{boosting4}{boosting4}
  \pgfdeclareimage[interpolate=true,height=3cm]{boosting8}{boosting8}
  \pgfdeclareimage[interpolate=true,height=3cm]{boosting64}{boosting64}

  \begin{center}
    \begin{tabular}{cc}
      \pgfuseimage{boosting1} & \pgfuseimage{boosting4} \\
      \pgfuseimage{boosting8} & \pgfuseimage{boosting64}
    \end{tabular}
  \end{center}

\end{frame}

\begin{frame}{Using Boosting Weights to Increase Dataset Accuracy}
  
  \begin{itemize}
  \item Boosted classifiers focus \alert{almost exclusively} on examples with \alert{high weights}
    \begin{itemize}
    \item High-weighted examples can have 100 times more weight than average
    \item To improve the classifier, we only need to improve these examples (they effectively ignore the rest)
    \end{itemize}
  \item If an example has a high boosting weight, it either:
    \begin{itemize}
    \item Is a difficult (and so informative) example; or
    \item Is mis-tagged
    \end{itemize}
  \item Re-tag (carefully) those with high weights
  \item Can get a true positive rate $> 50\%$ with this technique
  \end{itemize}
  
\end{frame}

\begin{frame}{Choosing a Learning Algorithm}
  \begin{quote}
    I would never have thought in 2008 that we would be using decision trees as our algorithm of choice for cutting edge research! (Didier Guillevic, Ph.D. in ML, Idilia researcher, 2008)
  \end{quote}
  
  \begin{itemize}
  \item If the mapping is done well, the choice of algorithm is less crucial
  \item Meta-algorithms (bagging, boosting) 
  \item Practicality is the key
    \begin{itemize}
    \item Productivity $\propto$ modify-test cycle time
    \item Much more time spend \alert{improving} than \alert{building}
    \end{itemize}
  \end{itemize}
  
\end{frame}

\begin{frame}{Undesirable Characteristics of Classifiers}
\begin{itemize}
\item Can't run the same experiment twice...
  \begin{itemize}
    \item Results got worse when I retried (\alert{repeatability})
    \item It took 300 CPU-years and my account was revoked (\alert{speed})
  \end{itemize}
\item Black box that can't tell us how to improve (\alert{explorability})
\item Needs \$10 million of data to learn a non-planar decision surface (\alert{non-linearity})
\item Lots of knobs to play with
  \begin{itemize}
  \item No \alert{automatic capacity control} (done manually)
  \item Requires $\alpha_{0,0} ... \zeta_{23,37}$ to be set \emph{just right}... (\alert{automatic tuning})
  \item Need to perform manual \alert{feature selection} experiments
  \end{itemize}
\item Changing feature $z$ from $0.12345$ to $0.12346$ lead to 13\% better test performance (\alert{sensitivity})
\item Output is not \alert{probabilistic}
\end{itemize}
\end{frame}


%\subsection{Example: Word Sense Disambiguation}

\begin{frame}{Example: Word Sense Disambiguation}
  \begin{itemize}
  \item An important computational linguistics problem
    \begin{itemize}
      \item Needed to reliably solve information retrieval, machine translation, speech recognition, ...
    \end{itemize}
  \item Identify the meaning of words in context
    \begin{itemize}
      \item ``I enjoy a hot \alert{java} in the afternoon'' $\rightarrow$ \alert{coffee}
      \item ``The economy of \alert{Java} lags that of Indonesia'' $\rightarrow$ \alert{island}
      \item ``Weka is written in \alert{Java}'' $\rightarrow$ \alert{programming language}
      \item BUT: “Do \alert{Java} programmers in \alert{Java} drink \alert{java}?”
    \end{itemize}
  \end{itemize}
\end{frame}

\pgfdeclareimage[interpolate=true,height=1.4cm]{telescope1}{telescope1}
\pgfdeclareimage[interpolate=true,height=1.4cm]{telescope2}{telescope2}
\pgfdeclareimage[interpolate=true,height=1.4cm]{telescope3}{telescope3}
\pgfdeclareimage[interpolate=true,height=1.4cm]{telescope4}{telescope4}
\pgfdeclareimage[interpolate=true,height=1.4cm]{telescope5}{telescope5}
\pgfdeclareimage[interpolate=true,height=1.4cm]{telescope6}{telescope6}

%\subsection{Different Types of Ambiguity}

\begin{frame}{Different Types of Ambiguity}

  \begin{itemize}
  \item I saw the man with the telescope.\\
    \fbox{\pgfuseimage{telescope1}}\hskip 0.2cm
    \fbox{\pgfuseimage{telescope2}}\hskip 0.2cm
    \fbox{\pgfuseimage{telescope3}}\hskip 0.2cm
    \fbox{\pgfuseimage{telescope4}}
    
  \item I saw the cow with the telescope.\\
    \fbox{\pgfuseimage{telescope5}}\hskip 0.2cm
    \fbox{\pgfuseimage{telescope6}}
  \item Different types of ambiguity; different ways of resolving
  \item Sometimes not possible to know \alert{(ill-posed)}
  \item Some kinds of errors are very costly
  \end{itemize}
  
\end{frame}

\begin{frame}{Difficulties Particular to Computational Linguistics}

  \begin{itemize}
  \item Language is inherently ambiguous
  \item Meaning is difficult to represent precisely
  \item Difficulties of annotation
    \begin{itemize}
    \item Expert humans can be $< 80\%$ accurate
    \end{itemize}
  \item Uneven Cost of errors
    \begin{itemize}
    \item Some errors are catastrophic; others inconsequential
    \end{itemize}
  \item Ill-posedness
  \item Skewed towards common meanings
    \begin{itemize}
    \item Uncommon meanings are just as important
    \end{itemize}
  \end{itemize}

\end{frame}



\end{document}
